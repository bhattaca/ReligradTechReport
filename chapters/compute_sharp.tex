
% \begin{wrapfigure}{r}{0.5\linewidth}
%\centering
%\includegraphics[width=0.6\linewidth]{images/mergesharp.3.eps}

%\caption{Isosurface mesh constructed by {\sc MergeSharp}.
%    Red isosurface vertices are the sparse selected edge vertices. The inset shows all the edge vertices generated in green, the larger blue cube shows the neighborhood around the smaller blue internal cube, all the green vertices in this larger neighborhood are merged to the isosurface vertex generated by the interior blue cube.}

%\label{fig:cubeB}    \end{wrapfigure}

%\begin{figure}
%\centering
%\includegraphics[width=0.6\linewidth]{images/mergesharp.3.eps}
%
%\caption{Isosurface mesh constructed by {\sc MergeSharp}.
%Red isosurface vertices are the sparse selected edge vertices. The inset shows all the edge vertices generated in green, the larger blue cube shows the neighborhood around the smaller blue internal cube, all the green vertices in this larger neighborhood are merged to the isosurface vertex generated by the interior blue cube.}
%
%\label{fig:cubeB}
%\end{figure}

\section{Computing Points on Sharp Features}
\label{sec:computeSharpPoints}

Once we have reliable gradients,
we can use those gradients to compute the location of isosurface vertices 
on sharp edges and corners.
The gradients can also be used to compute the location 
of isosurface vertices on smooth regions of the isosurface.
The algorithm for computing isosurface vertices from gradients
is given in~\cite{bw-cisec-13}.
It is a modification of Lindstrom's algorithm in~\cite{l-oslpm-00},
with surface normals replaced by gradients.

Let $g_i$ and $s_i$ be the gradient and scalar value, respectively,
at point $p_i$.
Let $\sigma$ be the isovalue.
The set $h_i = \{x : g_i \cdot (x-p_i) + s_i = \sigma \}$ is a plane in 3D.
Equivalently, plane $h_i$ is
$\{x : g_i \cdot x = \sigma - (g_i \cdot p_i + s_i) \}$.

Given a set $\{(p_i,g_i,s_i)\}$ of $k$ points and their associated
gradients and scalar values,
define a matrix $M$ whose $i$'th row is $g_i/|g_i|$
and a column vector $b$ whose $i$'th element is 
$(\sigma - (g_i \cdot p_i + s_i))/|g_i|$.
(We divide by $|g_i|$ so that all normal directions have equal weight.)
This gives a set of $k$ equations $Mx = b$
where $M$ is a $k\times3$ matrix and $x$ and $b$ are column vectors
of length $k$.
In general, this system is over-determined so we wish to find
the least squares solution.

We use the singular valued decomposition (SVD) of $M^T M$
to find an approximate solution $x^*$ to $M x = b$.
We use the number of large singular values of $A$
to determine whether $x^*$ is on a sharp corner,
a sharp edge or a smooth region on the surface.
Details are in Appendix~\ref{appendix:Lindstrom}.
