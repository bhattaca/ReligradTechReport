% Selecting sharp vertices

\section{Sparsifying Isosurface Vertices}
\label{sec:select_sharp}

The algorithm in the previous section produces one isosurface vertex
for each grid cube intersected by the isosurface.
Points on sharp edges and corners are identified 
as such by the algorithm.
However, isosurface vertices on sharp features
may lie very close together.
If those vertices are used for mesh generation
or are joined together to form sharp curves,
they need to sparsified so there is spacing between them.
Fortunately, the grid structure makes this easy.
We use a procedure from~\cite{bw-cisec-13}
to sparsify the set of isosurface vertices on sharp features.

Let $S$ be the list of isosurface vertices on sharp features
sorted by increasing distance from the grid cube centers.
Select the first vertex $v$ in $S$.
Let $\cb$ be the cube containing $v$.
Delete any vertices of $S$ which lie in $\cb$ 
or in any of the 26 grid cubes which share a vertex with $\cb$.
Repeat until $S$ contains no more vertices.
If all grid edges have length $L$,
then the resulting isosurface vertices are never closer
than distance $L$.
Figure~\subref*{fig:setA.crop1.mesh.2} shows the original sharp edge vertices and figure~\subref*{fig:setA.crop1.mesh.3} shows the sparse set for an edge in a real CT dataset.




