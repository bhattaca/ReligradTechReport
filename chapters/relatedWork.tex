
\section{Related Work}

Algorithms for constructing isosurfaces with sharp features
are given
in~\cite{ab-fpmmo-03,hwco-cmsaf-05,jlsw-dchd-02,kbsh-fssev-01,
ms-ispmg-10,sw-dcss-02,sw-dmcpc-04,Varadhan:2003:fss,zhk-dctps-04}.
All these algorithms require
exact surface normals as part of the input.

\MergeSharp~\cite{bw-cisec-13} is an algorithm
for constructing isosurfaces with sharp features from gradient data.
Input to the algorithm is gradient grid data,
scalar values and gradient vectors at each vertex of a regular grid.
The algorithm can handle noise in the gradient vectors 
and missing gradient vectors.

Salman et al.~\cite{sym-fpmg-10} and Dey et al.~\cite{dgqsww-fprss-12}
reconstructed piecewise smooth surfaces with sharp features 
from point cloud data
by separately placing mesh vertices on the sharp features
and vertices inside the smooth patches.
They place ``protecting ball'' around mesh vertices on sharp features
so that no vertices inside the smooth patches are placed 
near the sharp features.
Algorithm \MergeSharp does something similar,
merging grid cubes around sharp features so that isosurface vertices
on sharp features are ``isolated'' away from other vertices.

Fleishman et al.~\cite{fcs-rmlsf-2005} introduced a least-squares technique to reconstruct a piecewise smooth surface. The sharp features are reconstructed as intersection of these smooth regions. Oztireli et al.~\cite{Oeztireli2009} extended the moving least square reconstruction to sharp features using kernel regression. The strength of robust kernel regression, makes this method robust to noise. Avron et al.~\cite{avron2010L} used a l1-sparse approach to reconstruct sharp features from point set.

Formulas for improving the numerical accuracy of gradient computations
from scalar data
are given in~\cite{aml-ger-10,ham-thqge-11,mmmy-cnes-97}.
These formulas assume the gradient vector field is smooth
and do not work when there are discontinuities in the vector field.
They also do not work when there is noise in the input scalar data.

Anisotropic diffusion is a technique by which the filtering
of surface normals or field gradients changes based on local curvature.
Gradients or normals in low curvature regions are moved to agree 
with their neighbors.
Gradients or normals in high curvature regions are moved only slightly.
Anisotropic diffusion for mesh smoothing is described
in~\cite{bx-adsfs-03,cdr-agdsp-00,twbo-gssad-02,twbo-gspnm-03}.
Tasziden et. al.~\cite{tw-adsnf-03} used anisotropic diffusion 
to preserve features in isosurface reconstruction.

Features in papers on anisotropic diffusion are high curvature regions,
not regions with normal or gradient discontinuities (infinite curvature.)
Anisotropic diffusion applied to surfaces or gradient fields
with discontinuities will filter noise from smooth regions,
but it will not improve estimations at discontinuities
or assist in identifying such discontinuities.

